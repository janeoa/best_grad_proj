\documentclass[conference]{IEEEtran}
\IEEEoverridecommandlockouts
% The preceding line is only needed to identify funding in the first footnote. If that is unneeded, please comment it out.
\usepackage{cite}
\usepackage{amsmath,amssymb,amsfonts}
\usepackage{algorithmic}
\usepackage{graphicx}
\usepackage{textcomp}
\usepackage{xcolor}
\def\BibTeX{{\rm B\kern-.05em{\sc i\kern-.025em b}\kern-.08em
    T\kern-.1667em\lower.7ex\hbox{E}\kern-.125emX}}
\begin{document}

\title{Data-Driven Control for DC Motor
}

\author{\IEEEauthorblockN{Oraz Ospanov}
\IEEEauthorblockA{
oraz.ospanov@nu.edu.kz}
\and
\IEEEauthorblockN{Asset Malik}
\IEEEauthorblockA{
asset.malik@nu.edu.kz}
\and
\IEEEauthorblockN{Andrey Yershov}
\IEEEauthorblockA{
andrey.yershov@nu.edu.kz}
}

\maketitle

\begin{abstract}
Data-Driven control techniques have become increasingly popular in recent years due to the wide availability of data and progress in data science.
Data-Driven control design methods bypass the system identification
step and directly exploit collected data to construct the controller.
In this paper, we investigate the application of data-driven
methods to the control of DC motor drives.
\end{abstract}

\begin{IEEEkeywords}
data-driven control, DC motor
\end{IEEEkeywords}

\section{Literature overview}

The article \cite{silva2019a} utilizes Data-Driven Control to train the control gains for better system controllability.

The article \cite{naung2018a} presents Data-Driven Control used for system parameters estimation, an approach that we are planning to use.

\cite{carlet2020data} is an important article for our project, as it explores predictive current control for
synchronous motor drives - a topic closely related to ours. 

\cite{hanke2019continuous} introduces Permanent Magnet Synchronous Motor ripple avoidance using DDC.

\cite{coulson2019data} introduces novel data-enabled predictive
control (DeePC) algorithm is presented that computes optimal
and safe control policies using real-time feedback driving the
unknown system along a desired trajectory while satisfying
system constraints.

\cite{hou2013model} is a canonical survey paper in the field. The authors explore the recent advances in DDC and discuss the perspectives of future related work.

Article \cite{de2019a} discusses the important attributes for systems with data-driven control: Stabilization, Optimality, and Robustness.

Article \cite{rosolia2018a} explores the ability of the data-driven system to adapt and thus improve its performance in case of changes of parameters.

Article \cite{bu2018a} analyzes the behavior of data-driven control in the extreme case of output saturation. 

Article \cite{pravallika2021a} is an official MATLAB tutorial to DDC, which we follow with our designed controller simulation in the beginning of our project.

A recent article \cite{Berberich_2020} overviews a problem of state-feedback controllers for discrete-time linear time-invariant systems, based directly on measured data. 

A PhD thesis \cite{kergus} explores a Data-driven model reference control in the frequency-domain. 

An article \cite{CAMPESTRINI20172628} deals with Data-Driven (DD) control design in a Model Reference (MR) framework. 

Article \cite{vuillemin2019hybrid} shows a discrete-time control-law from frequency-data of a continuous-time plant so that their hybrid interconnection matches a given continuous-time reference model up to the Nyquist
frequency. 

\section{Introduction}

\section{Methods}

\section{Results}

\bibliographystyle{plain}
\bibliography{bibliography.bib}

\end{document}
